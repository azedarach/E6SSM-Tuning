\documentclass[12pt]{article}
\usepackage{fullpage}
\begin{document}
\noindent Dear editor,
\\

\noindent Thank you for sending us the referee report.  We would like to
sincerely thank the referee for their very helpful comments and questions
about our paper.  In response to each of these we have made appropriate
changes to our manuscript and feel these have improved the paper.  We
therefore hope that our paper will now be considered acceptable for
publication in Physical Review D.
\\
\noindent We respond to each point raised by the referee below, explaining
where appropriate what changes we have made in the paper to address the
referee's concerns.

\begin{itemize}
\item {\it ``in eq. (9) an additional parameter $\mu'$ appears.  Where
does it come from?  Even though it is later fixed to 5 TeV, it is not
clear if it has any impact on their study?  Even if it has no impact,
this should be stated somewhere.''}

The referee is correct to point out that the bilinear coupling $\mu'$
is absent from the list of superpotential terms, and is lacking an
explanation in our original submission. The $\mu'$ term is associated
with the incomplete $27'$ and $\overline{27}'$ to which the fields
$\hat{H'}$ and $\hat{\overline{H}'}$ belong, but is invariant under
the Standard Model gauge group and the low energy $U(1)'$
symmetry.  Additionally, because this parameter does not appear in the
EWSB conditions the impact from it on the fine tuning can be
neglected, and so fixing the value of $\mu'$ does not impact the
results.

To address this, we have added a short discussion immediately
following Eq.~(9) that explains the appearance of the additional parameter
and states that the chosen value of $\mu'$ does not impact on the fine
tuning in the study.

\item {\it ``page 11: the authors state that they neglect gauge kinetic
mixing.  In view of possible effects on the LHC bounds: is this really
justified by the size of the off-diagonal entries in the anomalous
dimension matrix of the gauge fields?  This can be in particular important
because gauge kinetic mixing and additional decays in supersymmetric
particles affect the mass bound on a Z obtained from LHC and can lower them
by up to 200 GeV as has been discussed in arXiv:0909.1320, arXiv:1107.1133,
arXiv:1205.5780, arXiv:1206.3513 or their ref. 102.  How would such a shift
affect their conclusions?''}

This is an important point which we neglected to comment on in our
original article.  In the models we investigate numerically in our
paper, if gauge kinetic mixing is absent at the GUT scale the
radiatively induced gauge kinetic mixing will be rather small at the
EW scale as has been shown in the literature.  Therefore we neglected
this.  However the referee is correct that it is not the case in all
$U(1)$ extensions and this may have a significant impact on the
$Z^\prime$ mass limit.

To address this issue we have added remarks to the end of section III
explaining why it is small in the models we consider, with reference
to the previous literature and comment on how it will impact on tuning
results if one considers a case with large gauge kinetic mixing.

\item {\it ``page 13: the author state that the include leading
two-loop contributions to the Higgs mass.  Which are those?  Are the
corresponding contributions to the minimization equations 16 included
and, thus, in their measure of fine-tuning?  Concerning the two-loop part:
in eq. (9) the couplings $\kappa_i S D_i D_i$ are included.  As the extra
D-quarks have to be heavy, one would naively expect that they give a
sizable contribution to the Higgs masses at the 2-loop level, in particular
as they take in their numerical studies $|\lambda| \leq 3$.  Are those
included as well?''}

The referee is correct that we missed the references specifying what
we mean by the leading two-loop corrections, so this was unclear.  We
included leading two-loop corrections calculated using effective field
theory methods, which have been generalized from well known
calculations in the MSSM and NMSSM.

To address this issue we have added references pointing out where we
obtained the leading two-loop Higgs corrections from.  For
completeness we also added a footnote pointing readers to the new tool
in the SARAH package which calculates the two-loop corrections to
Higgs masses in non-SUSY models.

We also understand why the referee may expect $\kappa$ couplings to give
a large contribution to the light Higgs mass, however we find that already
at one-loop the contribution from this coupling is small. The main reason
for this is the fact that singlet mixing with lightest Higgs is highly
suppressed in these models, which is a result of the large singlet VEV
required to meet the $Z^\prime$ limit.  The impact of the $\kappa$
couplings on the Higgs mass was also looked at in section 5.2.2 of
hep-ph/0510419.

The leading two-loop corrections to the effective potential are not
included in our calculation of the fine tuning.  Although the one-loop
effective potential corrections are important they are already smaller
than the RGE corrections and so we do not expect the two-loop
effective potential corrections to contribute at the same level as
the terms we include nor one which would significantly alter our
conclusions.

To address this we have added a comment specifying the precise one
loop corrections we use to the end of the paragraph below Eq.~(30).

\item {\it ``on page 15 the authors claim that for $M_X = 10^{16}$ GeV a
lower bound of 200 for the fine tuning taking $m_h = 125$~GeV. Looking
at their approximation in the calculation of $m_h$ this mass has most
likely a theory uncertainty of about 10 GeV and, thus, I do not
understand how they can make such a statement.  This also concerns
their statement on page 17.''}

The referee is correct to point out that there is a large theoretical
uncertainty on the Higgs mass and we did not mean to suggest
otherwise.  When we discuss the fine tuning dependence on the Higgs
mass we are trying to show the qualitative behavior of the tuning
increasing with the Higgs mass and how this changes depending on the
UV cutoff, using the numbers we quote to illustrate this.  We agree
that the theoretical error on the Higgs mass is rather large and that
this affects the impact the Higgs mass measurement has on the minimum
tuning that is consistent with observation.  However such discussions
raise difficult questions about how large the theoretical error is and
how to address that in the context of fine tuning.  Therefore we leave
such tricky questions to papers which are focussed on the impact on
the Higgs rather than the fine tuning from the $Z^\prime$ mass which
we are interested in here.

To address the referee's remarks we have adjusted the wording of the
first comment referred to and have added a footnote pointing out that
there will be a sizable uncertainty of the Higgs mass.

\item {\it ``looking at their table II, one sees that in certain parts of
the parameter space, $A_t$ can be much larger than the corresponding soft
masses. This implies the danger of color and charge breaking minima as is
known for long time. I guess that quite some points are excluded by this.
Moreover, it seems that they allow also $|\mu|$ to be below 100 GeV which
would conflict LEP2 data. Are such points really included in their analysis?''}

For parts of the parameter space scanned the referee is correct to
point out that there is a danger of color or charge breaking minima,
and that the range of values for $\mu$ scanned over includes values
that would be in conflict with experimental limits.  In our analysis
we excluded points that may lead to such minima by removing those
which have tachyons in the spectrum.  This however was not commented on
in our submission.  Regarding the LEP2 data and chargino mass bounds,
in Figures 2 and 4 the chargino mass limit was not imposed.  This is
because the LEP bounds are not severe enough to affect the minimum
obtainable fine tuning in the models we looked at and we wanted to
consider the impact of the chargino mass separately in Figure 3.  There
we show that much higher limits set at the LHC can have a meaningful
impact.  Nevertheless the referee is correct in pointing out that this should
be addressed in the text.

To address these points, a short comment has been added on page 17,
prior to Table II, in which the theoretical constraints that are applied to
the analysis are clarified.  In particular, we explain that in order to
exclude points that may have color and charge breaking minima we only allow
points that do not lead to tachyonic sparticle masses.  In addition to this,
we have added a remark on page 18 to point out that imposing the LEP bound
does not have an effect on the results, but that LHC limits can have a
significant impact.

\item {\it ``page 6: it is stated, that the $\hat N$ field remains
interactionless. However, to obtain light neutrino masses, there should be
at least Yukawa interactions.''}

The referee is correct to point out this out, and the sentence should instead
state that the $\hat N$ field does not participate in the gauge interactions
under the choice of $U(1)_N$. We have corrected this to address the referee's
comments.

\item {\it ``page 7: the $\hat N$ field is missing in equation 4''}

The referee's comment indicates that it was unclear the equation describes
only the low energy matter content in each $27$-plet, once the heavy fields
have been integrated out. The right-handed neutrinos are assumed to gain
large masses at some intermediate scale above the electroweak scale in order
to facilitate a see-saw mechanism for generating light neutrino masses.
Consequently they do not appear in the low energy matter content, so that
the equation is correct as currently written.

In order to address this and clarify the discussion, we have added a comment
immediately before the equation that explicitly mentions that the right-handed
neutrinos have been integrated out of the low energy theory.

\item {\it ``the journal is missing in ref. 57''}

This has been corrected by adding the journal title.


\end{itemize}

In addition to the changes to address the referee's comments we also took the
opportunity to add two references we had missed out when we submitted the
article to PRD which are:
\begin{itemize}
\item [19]  M.~Drees, N.~K.~Falck and M.~Gluck, Phys.\ Lett.\ B {\bf 167},
187 (1986).
\item [46]  C.~Boehm, P.~S.~B.~Dev, A.~Mazumdar and E.~Pukartas, JHEP
{\bf 1306}, 113 (2013)
\end{itemize}

\end{document}
