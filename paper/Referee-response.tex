\documentclass[12pt]{article}
\usepackage{fullpage}
\begin{document}
\noindent Dear editor,
\\

\noindent Thanks for sending us the referee responses. We would like to thank
the referee for their helpful comments and questions regarding our
paper. In response we have made changes to our manuscript and feel
these have improved the paper, we therefore hope our paper will now be
accepted for publication in PRD.
\\
\noindent We respond to each point rasied by the referee below, exaplaining
where appropriate what we have done to address it in the paper.


\begin{itemize}
\item Dylan to addresss

\item {\it ``page 11: the authors state that they
  neglect gauge kinetic mixing. In view of possible effects on the LHC
  bounds: is this really justified by the size of the off-diagonal
  entries in the anomalous dimension matrix of the gauge fields? This
  can be in particular important because gauge kinetic mixing and
  additional decays in supersymmetric particles affect the mass bound
  on a Z obtained from LHC and can lower them by up to 200 GeV as has
  been discussed in arXiv:0909.1320, arXiv:1107.1133, arXiv:1205.5780,
  arXiv:1206.3513 or their ref. 102. How would such a shift affect
  their conclusions?''}
 
This is an important point which we neglected to comment on in our
origional article. In the models we investigate numerically in our
paper, if gauge kinetic mixing is absent at the GUT scale it, the
radiatively induced gauge kinetic mixing will be rather small at the
EW scale as has been shown in the literature.  Therefore we neglected
this. However the referee is correct that it is not the case in all
$U(1)$ extensions and this may have a significant impact on the
$Z^\prime$ mass limit.

To address this issue we have added remarks to the end of section III
explaining why it is small in the models we consider, with referemce
to the previous literature and comment on how it will impact on tuning
results if one considers a case with large gauge kinetic mixing.

\item {\it ``page 13: the author state that the include leading
  two-loop contribu- tions to the Higgs mass. Which are those? Are the
  corresponding contributions to the minimization equations 16
  included and, thus, in their measure of fine-tuning? Concerning the
  two-loop part: in eq.(9) the couplings κi SDi Di are included. As
  the extra D-quarks have to be heavy, one would naively expect that
  they give a sizable contribution to the Higgs masses at the 2-loop
  level, in particular as they take in their numerical studies |λ| ≤
  3. Are those included as well?''}

The referee is correct that we missed the references specifying what
we mean by the leading two-loop corrections, so this was unclear. We
included leading two-loop corrections calculated using effective field
theory methods, which have been generalised from well known
calculations in the MSSM and NMSSM. 

To address this issue we have added references pointing out where we
obtained the leading two-loop Higgs corrections from.  For
completeness we also added a footnote pointing readers to the new tool
in the SARAH package which calculates the two-loop corrections to
Higgs masses in non-SUSY models.

We also understand why the referee
may expect $\kappa$ couplings to give a large contribution to the
light Higgs mass, however we find that already at one-loop the
contribution from this coupling is small. The main reason for this is
the fact that singlet mixing with lightest Higgs is highly suppressed
in these models, which is a result of the of the large singlet VEV
required to meet the $Z^\prime$ limit.  The impact of the $\kappa$
couplings on the Higgs mass was also looked at in section 5.2.2 of
hep-ph/0510419.

The leading two loop corrcetions to the effective potential are not
included in our calculation of the fine tuning.  Although the one loop
effective potential corrections are important they are already smaller
than the RGE corrections and so we do not expect the two loop
effective potential corrections to contribute at to the same level as
the terms we include nor one which would significantly alter our
conculusions.

To address this we have added a cooment specifying the precise one
loop corrections we use to the end of the paragraph below Eqn 30.

 


\item {\it ``on page 15 the authors claim that for MX = 1016 GeV a
  lower bound of 200 for the fine tuning taking mh = 125 GeV. Looking
  at their approximation in the calculation of mh this mass has most
  likely a theory uncertainty of about 10 GeV and, thus, I do not
  understand how they can make such a statement. This also concerns
  their statement on page 17.''}  

The referee is correct to point out that there is a large theoretical
uncertainty on the Higgs mass and we did not mean to suggest
otherwise.  When we discuss the fine tuning dependence on the Higgs
mass we are trying to show the qualitative behavior of the tuning
increasing with the Higgs mass and how this changes depending on the
UV cuttoff. We agree that the theoretical error on the Higgs mass is
rather large and that this affects the impact the Higgs mass
measurement has on the minimum tuning that is consistent with
observation.  However such discussions raise difficult questions about
how large the theoretcial error and how to address that in the context
of fine tuning. Therefore we leave such tricky questions to papers
which are focussed on the impact on the Higgs rather than the fine
tuning from the $Z^\prime$ mass which we are interested in here.

To address the referees remarks we have adjusted the wording of the
first comment he refers to and have added a footnote pointing out that
there will be a sizable uncertainty of the Higgs mass.

\item Dylan to address

\item Dylan to address
\item Dylan to address
\item Dylan to address



\end{itemize}

In addition to the changes to address the referees comments we also took the opportuinity to add two references we had missed out when we submitted the article to PRD which are:
\begin{itemize}
\item [19]  M.~Drees, N.~K.~Falck and M.~Gluck,Phys.\ Lett.\ B {\bf 167}, 187 (1986).
\item [46]  C.~Boehm, P.~S.~B.~Dev, A.~Mazumdar and E.~Pukartas, JHEP {\bf 1306}, 113 (2013)
 \end{itemize}



 




\end{document}
